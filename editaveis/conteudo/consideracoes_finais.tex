\chapter[Considerações Finais]{Considerações Finais}

Ao decorrer da disciplina, o conteúdo dado durante as aulas e a prática do desenvolvimento do projeto final, evidenciaram o papel importante da área de IHC dentro da engenharia de software. Os estudos do comportamento dos usuários em relação a utilização de um sistema são importantes para validar e verificar como o usuário se sente melhor durante a utilização deste sistema, e assim, poder implementar algo que forneça ao usuário uma ótima experiência ao se utilizar este sistema. 

Neste projeto, buscamos alcançar esta melhor experiência para o usuário, utilizando o conteúdo abordado em sala de aula na prática. Com o uso de protótipos, foram desenvolvidos as primeiras versão do nosso sistema para a então aplicação dos testes com os usuários. Foi então medido suas experiências durante a utilização do sistema por meio de questionários e da observação dos usuários durante a prática da prototipação, além dos feedbacks fornecidos pelos mesmos. E, baseado nessas avaliações criamos outros protótipos com o intuito de aperfeiçoar nossa ideia principal a medida que forem feitas as avaliações sobre esses objetos. Ao final, após se obter os feedbacks dos usuários sobre os protótipos, iniciamos a implementação do sistema real.

Durante o processo de criação dos protótipos houveram algumas mudanças, a maioria delas baseadas nas avaliações dos usuários como foi relatado, porém tivemos algumas dificuldades durante o desenvolvimento que nos impediram de implementar a ideia original, diminuindo, o então escopo inicial deste projeto. Após a avaliação dos questionários aplicados, foi detectado que houve uma melhora significativa no âmbito da usabilidade, porem não conseguimos atingir na prática a evolução dos resultados esperados, limitando novamente as próprias funcionalidades do sistema.
 
Portanto, concluímos que, apesar das dificuldades que apareceram durante o decorrer do projeto nos forçando a diminuir o escopo do projeto, alcançamos as metas definidas para garantirmos a boa experiência dos usuários ao utilizarem o nosso sistema.
\chapter[Metas Decorrente da Experiência do Usuário]{Metas de Decorrente da Experiência do Usuário}

O objetivo de desenvolver produtos interativos agradáveis, divertidos, esteticamente apreciáveis, etc, está principalmente na experiência que estes proporcionarão ao usuário. Neste contexto as metas decorrentes da experiência do usuário são dados subjetivos apoiados na satisfação de uso,  isto é, como o usuário se sentirá na interação com o sistema \cite{SHROPR}. Alguns destes valores subjetivos estão descritos nessa secção:

\hspace{1.3cm}
\textbf{Agradável:} o sistema deve possuir uma interface que satisfaça o gosto do usuário e que não o incomode.

\hspace{1.3cm}
\textbf{Esteticamente Apreciável:} o sistema deve possuir uma interface que o usuário note como sendo apropriada para sua utilização.

\hspace{1.3cm}
\textbf{Interessante:} o sistema deve fornecer um ambiente que instigue o usuário a ter interesse em descobrir e usar todas as funcionalidades disponíveis.

\hspace{1.3cm}
\textbf{Motivador:} o sistema deve influenciar na tomada de decisões do usuário, o motivando a navegar no mesmo.

\hspace{1.3cm}
\textbf{Proveitoso:} o sistema deve ter funcionalidades eficientes que não façam o usuário utilizar o aplicativo de forma desnecessária.

\hspace{1.3cm}
\textbf{Satisfatório:} o sistema deve ter funcionalidades, opções e personalizações que satisfaçam a todas as possíveis necessidades do usuário, evitando que ele se questione o porquê de não ser possível realizar determinada ação.

\section{Metas Escolhidas}

Com base nessas metas, foram definidas quais delas deveriam ser atingidas durante esse projeto, são elas, \textbf{agradável} e \textbf{proveitoso}, respectivamente, o sistema deve possuir uma interface que satisfaça o gosto do usuário e que não o incomode e o sistema deve ter funcionalidades eficientes que não façam o usuário utilizar o aplicativo de forma desnecessária.
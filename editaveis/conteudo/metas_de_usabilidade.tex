\chapter[Metas de Usabilidade]{Metas de Usabilidade}

\noindent
Com a finalidade de proporcionar a melhor experiência possível para os usuários, este sistema deverá atender as metas de usabilidade propostas por~\cite{Nielsen}.

\vspace{0.2cm}
\textbf{Eficácia:}

\hspace{1.3cm}
\textbf{Design estético:} As informações apresentadas ao usuário deverão sempre evitar possuir conteúdo que seja irrelevante ou raramente utilizado. Neste caso o sistema não basta só ser funcional, tem que ser esteticamente agradável.

\vspace{0.4cm}
\textbf{Eficiência:}

\hspace{1.3cm}
\textbf{Visibilidade do sistema:} Sinais e mensagens simples que podem ajudar muito na interação como exemplo, “carregando”, “atualizado”, “salvo” e outras que podem ajudar o usuário a saber onde está e o que fazer a seguir. Para isso, o sistema irá fornecer feedbacks apropriados em determinados momentos.

\hspace{1.3cm}
\textbf{Flexibilidade e eficiência no uso:} As pessoas pensam, sentem e agem de modo diferente, portanto permitir certa flexibilidade de uso pode ajudar muito o usuário a cumprir suas tarefas. Quando possível, serão disponibilizados atalhos para ações do usuário, dentro do sistema, com o intuito de que os usuários mais experientes possam executar suas tarefas mais rapidamente. 

\hspace{1.3cm}
\textbf{Diagnóstico e recuperação de erros:} Suporte para o usuário reconhecer, diagnosticar e recuperar erros. O sistema possuirá meios e avisos para o usuário conseguir gerenciar os erros ampliam tanto a eficiência quanto a própria utilidade da interface. Será usada, quando necessária, numa linguagem simples e construtiva para descrever a natureza de um problema e sugerir uma forma de solucioná-lo.

\vspace{0.4cm}
\textbf{Segurança:}

\hspace{1.3cm}
\textbf{Prevenção de erros:} Prevenir os possíveis erros que o usuário pode cometer interagindo com a interface oferecendo meios para ele se recuperar, isto pode ampliar a credibilidade que este usuário terá do sistema.

\vspace{0.4cm}
\textbf{Aprendizagem:}

\hspace{1.3cm}
\textbf{Mapeamento entre o sistema e o mundo real:} O sistema deverá se comunicar com o usuário de uma forma que ele possa compreender as     mensagens e informações sem ambiguidade ou de forma não totalmente  clara.

\hspace{1.3cm}
\textbf{Ajuda e documentação:} Por mais bem projetado que seja a solução pode ser que algo não pareça tão simples, o sistema fornecerá um painel de ajuda e uma documentação organizada para auxílio de todos os envolvidos.

\hspace{1.3cm}
\textbf{Reconhecer ao invés de lembrar:} Essa meta tem como objetivo a facilidade de uso do sistema. Ou seja, deve se evitar que o usuário faça várias ações para concluir seu objetivo. Fazer os elementos reconhecíeis contribui tanto para eficiência, quanto na eficácia ou mesmo no aprendizado do usuário, ou seja, fazer um botão que parece botão e evitar o oposto.

\vspace{0.4cm}
\textbf{Memorização:}

\hspace{1.3cm}
\textbf{Consistência e padrões:} Tanto para ajudar na memorização, quanto no aprendizado ou mesmo na sensação de segurança, o sistema irá manter a consistência e os padrões são cruciais para o usuário. Sejam eles para pop-ups, botões de confirmação ou cancelamento, etc. No intuito de manter o mesmo comportamento e aparência em todo o sistema.
	
\vspace{0.4cm}
\textbf{Utilidade:}

\hspace{1.3cm}
\textbf{Liberdade e controle ao usuário:} O sistema garantirá que o usuário aja livremente, pois pode ser a melhor forma para ele conseguir realizar sua tarefa. Apesar do sistema não possuir meios que possibilite o usuário à personalizar sua interface, a mesma será disposta afim de possibilitar está liberdade ao usuário.


\section{Metas de Usabilidade Escolhida}

Tendo em vista que estes princípios definidos por \cite{Nielsen} são de um ponto de vista mais abrangente, definimos que em nosso projeto final serão aplicados no produto final os princípios de aprendizagem e eficiência, sendo estes respectivamente \textbf{reconhecer ao invés de lembrar} e \textbf{flexibilidade e eficiência no uso}. Estes princípios foram escolhidos por se encaixarem de maneira adequada na nossa proposta, pois representam partes essenciais para aplicações de cunho informativo de rápido e fácil acesso.

As metas de usabilidade definidas para este projeto serão alcançadas a partir do \textit{feedback} dos usuários.
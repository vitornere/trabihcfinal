\chapter[Introdução]{Introdução}

O Curta Mais é um site de busca de entretenimento em geral na cidade de Goiânia - GO, com a finalidade de dar uma identidade própria à capital goiana, tendo em vista  que algumas pessoas de outras regiões demonstravam visões equivocadas, simplistas e até preconceituosas sobre a cidade, que hoje é uma capital próspera em constante desenvolvimento, com relevância nacional, estrutura para receber os mais variados tipos de eventos e um acervo cultural amplo, porém pouco aproveitado/conhecido pela população ~\cite{curtamais}.

Foi proposto como tema para ser abordado nesse projeto final da disciplina de IHC (Introdução Humano-Computador) a seleção de uma categoria de busca do site Curta Mais para criação de uma aplicação funcional. 

Sendo assim, trabalharemos em cima de uma categoria original do site Curta Mais, a categoria de Gastronomia, definindo dentro da mesma, algumas novas subcategorias de que esboçam uma ideia geral do que os visitantes do site encontrarão ao navegar por essa porção do site. 

Este documento aborda algumas das práticas relacionadas a criação de um projeto aplicando os conceitos ensinados na disciplina de IHC da Faculdade do Gama, UnB (Universidade de Brasília), sendo estes a definição de um ciclo de vida, que conduzirá as práticas do projeto, storyboards e protótipos para possíveis avaliações e testes com usuários reais, a fim de testar nossas ideias com possíveis usuários do aplicativo, verificando se o aplicativo atende as necessidades e interesses dos utilizadores. Também estarão presentes neste documento as ferramentas utilizadas  para criação das práticas do projeto, além das ferramentas de comunicação do grupo idealizador do projeto.

Este projeto contará com a descrição do conjunto das metas de usabilidades definidas e requisitos funcionais e requisitos não-funcionais que deverão estar contidos no produto final gerado por este projeto. Além de estar documentado o planejamento feito ao longo do projeto, com o intuito de descrever a organização da nossa equipe quanto as tarefas relacionadas a está práticas e tarefas descritas neste documento.

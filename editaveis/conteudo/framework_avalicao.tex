\chapter[Framework de Avaliação]{Framework de Avaliação}

Dentro do conjunto de frameworks de avaliação existem vários paradigmas de avaliação que diferem uns dos outros nos aspectos de como são feitas as avaliação, em relação ao papel do usuário e onde são feitos os testes.
Fizemos a escolha a partir dos paradigmas de avaliação demonstrados na disciplina de IHC, sendo estes paradigmas:

\hspace{1.3cm}
\textbf{Rápido e sujo:} Descreve a prática comum na qual os designers informalmente obtem feedback dos usuários ou consultores para confirmar que suas ideias estão alinhados com as necessidades dos usuários.

\hspace{1.3cm}
\textbf{Teste de usabilidade:} Testes de usabilidade envolve a gravação de desempenho dos usuários típicos em tarefas típicas em ambientes controlados. Também pode ser usado observações de campo.

\hspace{1.3cm}
\textbf{Campos de estudo:} Os estudos de campo são feitas em ambientes naturais com o objetivo de compreender o que os usuários fazem naturalmente e como impactos tecnológicos afetam eles.

\hspace{1.3cm}
\textbf{Avaliação preditiva:} Especialistas aplicam seus conhecimentos de usuários típicos, muitas vezes guiadas por heurísticas, para prever problemas de usabilidade.


\subsection{Escolha do Framework de Avaliação}

Utilizaremos para avaliar a usabilidade de nosso sistema os paradigmas de usabilidade, rápido e sujo, e os testes de usabilidade, sendo que estas adaptam melhor ao nosso contexto.

A seleção do paradigma rápido e sujo se deve ao fato do retorno de feedbacks dos usuários serem rápidos, em relação a interface do sistema, podendo ser feita em qualquer local, onde o usuário sinta-se confortável. Mantendo sempre o controle do usuário, fazendo com que ele haja da maneira mais natural possível, sendo que esse é o foco da do paradigma, além de ser totalmente prático como ficou evidenciado. A escolha dos testes de usabilidade deve-se ao foco na observação do usuário, onde seriam utilizadas gravações de vídeo durante a aplicação dos testes, possibilitando assim uma validação mais consistente do feedback do usuário. Os testes de usabilidade é utilizado sobre a prototipagem do sistema, que também foi aproveitado para os testes rápidos e sujos do nosso projeto.

Dado que os testes aplicados com os usuários foram utilizados alguns protótipos do sistema em conjunto com os métodos de avaliação selecionados, onde ao final foram recolhidos os feedbacks dos usuários, foi possível avaliar se o sistema estava ou não alinhado com as necessidade do usuário.
